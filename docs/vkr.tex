% Created 2022-11-30 Ср 23:32
% Intended LaTeX compiler: pdflatex
\documentclass[PI, VKR]{HSEUniversity}
         \usepackage{array,tabularx,tabulary,booktabs,longtable,multirow}
         \Year{\the\year{}}
         \supervisor{к.т.н., доцент кафедры Информационных технологий в бизнесе НИУ ВШЭ-Пермь}{А. В. Бузмаков}
                  \Abstract{}


\usepackage[utf8]{inputenc}
\usepackage[T1]{fontenc}
\usepackage{graphicx}
\usepackage{longtable}
\usepackage{wrapfig}
\usepackage{rotating}
\usepackage[normalem]{ulem}
\usepackage{amsmath}
\usepackage{amssymb}
\usepackage{capt-of}
\usepackage{hyperref}
\author{Соломатин Роман Игоревич}
\date{\today}
\title{Разработка сайта для автоматического сбора, анализа и визуализации информации по этичности компаний}
\hypersetup{
 pdfauthor={Соломатин Роман Игоревич},
 pdftitle={Разработка сайта для автоматического сбора, анализа и визуализации информации по этичности компаний},
 pdfkeywords={},
 pdfsubject={},
 pdfcreator={Emacs 28.2 (Org mode 9.6)},
 pdflang={Ru}}
\usepackage{biblatex}
\addbibresource{/home/samoed/Desktop/ESGanalysis/docs/library.bib}
\begin{document}

\maketitle

\chapter*{Введение}
\label{sec:org9ca4251}
При работе с различными компаниями возникают проблемы их надежности, то как они ведут себя в спорных ситуациях, есть ли сервисы направленные на взаимодействие с клиентами, как ведут себя в спорных ситуациях.

В настоящее время не существуют сервисов, которые позволяют оценить этичность компании.

Предмет исследования – оценка этичности компаний.

Объект исследования – оценка этичности компаний с помощью нейронных сетей.

Цель работы – создание системы для оценки этичности компаний.

Исходя из поставленной цели, необходимо:

\begin{enumerate}
\item Провести анализ
\item Спроектировать систему
\item Реализовать систему
\item Провести тестирование системы
\end{enumerate}

Этап анализа должен:
\begin{enumerate}
\item Анализ предметной области
\item Анализ существующих алгоритмов
\end{enumerate}

Этап проектирования должен включать:
\begin{enumerate}
\item Проектирование серверной части
\item Проектирование модели для определения этичности
\item Проектирование клиентской части приложения
\end{enumerate}

Этап реализации должен включать:
\begin{enumerate}
\item Описание сбора данных
\item Реализации модели
\item Реализации серверной части
\item Реализации клиентской части
\end{enumerate}

Этап тестирования должен включать:
\begin{enumerate}
\item Тестирование модели
\item Тестирование серверной части
\item Тестирование клиентской части
\end{enumerate}
\chapter{Анализ предметной области}
\label{sec:org928e80b}
\chapter{Проектирование системы}
\label{sec:orgdc7e09b}
\chapter{Реализация системы}
\label{sec:org9ebcf24}
\chapter{Тестирование системы}
\label{sec:org6495ae3}
\chapter*{Заключение}
\label{sec:org138cced}
\putbibliography
\appendix
\end{document}
